% !TeX root = ../main.tex

\begin{acknowledgement}

%首先,我要感謝我的指導教授黎士瑋博士,您帶領我走過一遍碩士班的研究過程,讓我受益良多。
%您在研究期間的建議不僅帶給我研究上的進展,其蘊含的分析角度也常比我自己的想法還更全面且有條理,讓我評斷一件事情的時候有更完整的思考。
%在撰寫論文的這幾個月,您不厭其煩的替我審閱我的論文,指出我文字描述中的邏輯謬誤和解釋不清之處,使得我的學術寫作及邏輯闡述能力更加精進。
%
%本論文的完成亦得感謝擔任我口試委員的黃敬群 (jserv) 教授與蕭旭君教授,因為你們的建議與意見,使得我的論文能夠更完整且嚴謹。
%
%另外感謝江昱勳與杜展廷同學,沒有你們的貢獻與合作,KrustVM的研究與投稿論文不會順利地完成。
%
%我也要感謝我的家人和朋友們,有你們的支持我才能完成論文。
%
%此外,特別感謝 OpenAI 的 ChatGPT,它不厭其煩的幫助我優化了論文的英文表達,許多論文中的字句 (包含了本致謝) 得益於它才能夠如此通順,清晰和流暢。
%
%最後,謹以此文獻給逝去的歲月。

首先,我要衷心感謝我的指導教授黎士瑋博士。在研究期間,您的指導和建議讓我受益匪淺。
您在研究期間的建議不僅帶給我研究上的進展,其蘊含的分析角度也常比我自己的想法還更全面且有條理,讓我學到評斷一件事情更完整的思考方式。
在撰寫論文的過程中,您不辭辛勞地審閱我的稿件,指出其中的邏輯謬誤和解釋不清之處,使我在學術寫作和邏輯闡述方面精進了許多。

本論文的完成亦得感謝擔任我口試委員的黃敬群 (jserv) 教授與蕭旭君教授,您們的寶貴意見和建議使得我的論文更加完整和嚴謹。
此外,感謝江昱勳與杜展廷同學的合作和貢獻,沒有你們的協助, KrustVM 的研究和論文投稿將不會如此順利地完成。
我也要衷心感謝我的家人和朋友們,是你們的支持讓我能夠順利完成論文。

特別感謝 OpenAI 的 ChatGPT,它不厭其煩地幫助我優化論文的英文表達。
許多論文中的文字,包括本致謝部分,都得益於它,使得我的論文變得更加通順、清晰和流暢。

最後,謹以此文獻給逝去的歲月。
\begin{flushright}
章瑋麟\\
%國立臺灣大學資訊工程學系暨研究所\\
中華民國一百一十二年七月
\end{flushright}

\end{acknowledgement}
