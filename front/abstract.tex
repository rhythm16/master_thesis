% !TeX root = ../main.tex

\begin{abstract}

通用的虛擬機器監測器在雲端計算環境中發揮著至關重要的作用,它們負責監管虛擬機器的硬體資
源。然而,其日益複雜的設計和廣泛的攻擊面引發了重大的安全憂慮。攻擊者如果利用特
權虛擬機器監測器的漏洞,就能夠不受限制地存取虛擬機器中的數據,從而危及其資訊安全。以前嘗
試將虛擬機器監測器重構為小型受信任核心的嘗試存在局限性,因其安全性仍然依賴於受
信任核心的實現。此外,對TCB的形式化驗證需要大量的人力投入,難以適用於快速發
展的軟體專案。最近,由於其強大的記憶體安全保證和高性能,Rust語言的應用逐漸增加。
本論文着眼於解決將SeKVM中基於C語言的KVM(內核虛擬機器)TCB改寫並遷移到Rust的挑
戰,為此選擇了最近版本的Linux長期支持版本。通過這樣的改寫,我們實作出的虛擬機器監測器
KrustVM不僅能從最新的Linux進展中獲益,而且還能受益於Rust提供的安全保障。KrustVM
的設計重點在於最大化其不安全Rust程式碼的安全性。
我們將不安全程式碼與安全Rust隔離,並通過安全抽象將不安全
程式碼最小化。此外,利用Rust的型別系統,我們確保了受信任Rust核心進行的不安全記憶體
訪問的安全性。與KVM和SeKVM相比,KrustVM的性能損失不大,
展示了通過C到Rust的改寫來保障現有虛擬機器監測器的可行性。

\end{abstract}

\begin{abstract*}

Commodity hypervisors play a vital role in cloud computing environments by
overseeing hardware resources for virtual machines. However, their growing
complexity and extensive attack surface pose significant security concerns.
An attacker that exploits vulnerabilities in the privileged hypervisor
codebase can gain unfettered access to VM data, compromising their safety.
Previous attempts to retrofit hypervisors into small trusted cores have
limitations, as the security still relies on the implementation of the trusted
core. Moreover, formal verification on the TCB necessitates significant human
effort and is not easily applicable to rapidly evolving codebases.
Recently, Rust adoption has been increasing for its strong memory safety
guarantees and performance efficiency.
This thesis addresses challenges in rewriting and porting the C-based KVM TCB
in SeKVM to Rust for a recent Linux long term support version. This allows the
resulting hypervisor, \rustsec{}, to not only benefit from recent Linux
advancements, but also be protected by Rust's safety guarantees.
%Rust's memory safety guarantees only applies for safe Rust, as some safety
%rules are not enforced for unsafe Rust. However, this approach still provides
%benefits as the developer only need to audit the parts of the codebase that are
%implemented in unsafe Rust, effectively reducing the amount of code that
%requires manual review for memory safety bugs.
%Furthermore, the Rust-based implementation can be conveniently updated,
%as Rust conducts safety checks for safe Rust automatically.
%This thesis explores leveraging Rust to build a secure commodity hypervisor.
%We focus on rewriting SeKVM into KrustVM.
% SMALL CORE DESCRIPTION
%In \rustsec{}, a small trusted core written in Rust is incorporated to replace
%the C-based core of SeKVM, which serves to protect VM confidentiality and
%integrity.
% SMALL CORE DESCRIPTION END
%During the implementation of \rustsec{},
%we addressed challenges in linking a Rust TCB into Linux, rewriting SeKVM's
%C-based TCB in Rust, and bringing up \rustsec{} on real hardware.
%\rustsec{} incorporates
%\rustcore{}, a small trusted core written in Rust to protect VM confidentiality
%and integrity.
%To enhance the safety of the unsafe code in \rustsec{},
\rustsec{} is designed with a focus on maximizing the safety of its unsafe
Rust usages.
We minimized and separated unsafe code from safe Rust by enclosing unsafe code
within safe abstractions.
Additionally, Rust's type system is utilized to ensure the memory safety
of the unsafe memory accesses done by the trusted Rust core.
%[Then you focus on briefly disuss what you actually do -- enclosed unsafe code etc.]
%In addition,
%we minimized unsafe Rust usage, enclosed unsafe code within safe abstractions,
%and utilized Rust's type system to ensure the memory safety of unsafe
%operations in \rustcore{}.
\rustsec{} incurs modest overhead compared to mainline KVM and SeKVM, and
demonstrates the practicality of securing existing hypervisors through a
C-to-Rust rewrite.

\end{abstract*}
