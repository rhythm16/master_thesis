% !TeX root = ../main.tex

\begin{abstract}

中文摘要

\end{abstract}

\begin{abstract*}

Commodity hypervisors play a vital role in cloud computing environments by
overseeing hardware resources for virtual machines. However, their growing
complexity and extensive attack surface pose significant security concerns.
An attacker that exploits vulnerabilities in the privileged hypervisor
codebase can gain unfettered access to VM data, compromising their safety.
Previous attempts to retrofit hypervisors into small trusted cores have
limitations, as the security still relies on the implementation of the trusted
core. Moreover, formal verification on the TCB necessitates significant human
effort and is not easily applicable to rapidly evolving codebases.
Recently, Rust adoption has been increasing for its strong memory safety
guarantees and performance efficiency.
This thesis addresses challenges in rewriting and porting the C-based KVM TCB
in SeKVM to Rust for a recent Linux long term support version. This allows the
resulting hypervisor, \rustsec{}, to not only benefit from recent Linux
advancements, but also be protected by Rust's safety guarantees.
Furthermore, the Rust-based implementation can be conveniently updated,
as Rust conducts safety checks at compile-time automatically.
%This thesis explores leveraging Rust to build a secure commodity hypervisor.
%We focus on rewriting SeKVM into KrustVM.
In \rustsec{}, a small trusted core written in Rust is incorporated to replace
the C-based core of SeKVM, which serves to protect VM confidentiality and
integrity.
%During the implementation of \rustsec{},
%we addressed challenges in linking a Rust TCB into Linux, rewriting SeKVM's
%C-based TCB in Rust, and bringing up \rustsec{} on real hardware.
%\rustsec{} incorporates
%\rustcore{}, a small trusted core written in Rust to protect VM confidentiality
%and integrity.
In addition,
a modular design is adopted to secure the trusted Rust core. We separated and
minimized the unsafe Rust code from safe Rust by enclosing unsafe code within
safe abstractions, and utilized Rust's type system to ensure the memory safety
of the unsafe memory accesses done by the trusted Rust core.
%[Then you focus on briefly disuss what you actually do -- enclosed unsafe code etc.]
%In addition,
%we minimized unsafe Rust usage, enclosed unsafe code within safe abstractions,
%and utilized Rust's type system to ensure the memory safety of unsafe
%operations in \rustcore{}.
\rustsec{} incurs modest overhead compared to mainline KVM and SeKVM, and
demonstrates the practicality of securing existing hypervisors through a
C-to-Rust rewrite.

\end{abstract*}
