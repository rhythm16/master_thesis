% !TeX root = ../main.tex

\begin{abstract}

中文摘要

\end{abstract}

\begin{abstract*}

Commodity hypervisors play a vital role in cloud computing environments by
overseeing hardware resources for virtual machines. However, their growing
complexity and extensive attack surface pose significant security concerns.
An attacker that exploits vulnerabilities in the privileged hypervisor
codebase can gain unfettered access to VM data, compromising their safety.
Previous attempts to retrofit hypervisors into small trusted cores have
limitations, as the security still relies on the implementation of the trusted
core. Recently, Rust adoption has been increasing for its strong memory safety
guarantees and performance efficiency. Leveraging Rust, our work focuses on
rewriting SeKVM, a secure Linux KVM hypervisor, into \rustsec{}, the first
Rust-based secure Linux/KVM hypervisor. \rustsec{} incorporates
\rustcore{}, a small trusted core written in Rust to protect VM confidentiality
and integrity.
We addressed challenges in incorporating a Rust TCB into Linux, rewriting SeKVM's
C-based TCB in Rust, and bringing up \rustsec{} on real hardware. In addition,
we minimized unsafe Rust usage, enclosed unsafe code within safe abstractions,
and utilized Rust's type system to ensure the memory safety of unsafe
operations in \rustcore{}.
\rustsec{} suggests a modest overhead compared to mainline KVM and SeKVM, and
demonstrates the practicality of securing existing hypervisors through a
C-to-Rust rewrite.

\end{abstract*}
