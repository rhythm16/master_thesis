\chapter{Related Work and Future Work}
\label{sec:rwfw}

\section{Related Work}

\subsection{VM Protection}
%~\cite{lisosp21,zhang2011cloudvisor,zeyu20usenix,pkvm,fidelius-hpca18,sekvm}
Various previous work redesigned the hypervisor to protect VMs.
%Hardware and software approaches have both been explored, including nested
Cloudvisor \cite{zhang2011cloudvisor, zeyu20usenix} introduced a tiny security
monitor underneath the commodity hypervisor to protect the hosted VMs.
Twinvisor \cite{lisosp21} supports regular VMs and confidential VMs by running
hypervisors within both of ARM TrustZone's normal world and secure world.
HypSec and pkvm \cite{hypsec, pkvm} reduced the hypervisor's resource access
control component into to a small core to reduce the attack surface.
Unlike our work, none of them used Rust to secure their hypervisor
implementation.
\rustsec{} and SeKVM~\cite{sekvm} both leveraged HypSec's design~\cite{hypsec}
to retrofit and secure KVM, providing the same level of VM protection. SeKVM
included a formally verified core to protect VMs against an untrusted host Linux
kernel, while \rustsec{} relies on a Rust-based \rustcore{} to protect VMs.
Formal verification of the concurrent C-based SeKVM core requires significant
effort. The authors took two person-years to complete the correctness and
security proofs.
In contrast, our Rust-based implementation took less than one person-year.
Different from formal verification, because the Rust compiler automatically
ensures memory safety properties, our hypervisor codebase is flexible to
frequent updates.

\subsection{Rust-based Systems}
Recent work extended existing C/C++ systems with a Rust binding to enable
a Rust-based programming environment. Rust-SGX~\cite{rustsgx} and RusTEE
\cite{rustee} wrapped the C/C++ TEE SDK and exposed a safe Rust API
to enable Rust programming in TEE environments such as SGX and TrustZone.
Similarly, the Rust-for-Linux~\cite{Rust-for-Linux} project added
abstraction layers to the Linux kernel to facilitate Rust driver
programming with Rust.
Besides building a Rust binding, previous work re-implemented C-based
components in virtualization systems with Rust.
rust-vmm~\cite{rust-vmm} rewrote a subset of QEMU's functionalities
and separated them into libraries in Rust crates.
Firecracker~\cite{Firecracker}, crosvm~\cite{crosvm}, Cloud Hypervisor
\cite{CloudHypervisor}, and VMSH~\cite{VMSH} extended the
rust-vmm project with extra functionalities. These previous works
built on top of existing core systems. In contrast, our work
retrofitted Linux/KVM with a Rust-based TCB.
HyperEnclave~\cite{hyperenclave} relies on a Rust-based security
monitor to enforce isolation between enclave TEEs. Unlike
our work, the authors did not discuss the Rust monitor's implementation
and its unsafe Rust usage.

\section{Future Work}

Hardware features such as memory translation done by the Memory Management
Unit (MMU), and exception levels, are not modeled by Rust, therefore the
compiler is not able to check for misconfiguration or logical errors when
building software on top of these features.
In other words, potential logical bugs in \rustcore{} still undermines the
security of our hypervisor.
As an example, forgetting to clear page table entries in the host's NPT
makes way for a compromised host kernel to read the memory of a protected VM.
We can add one more layer of defense by taking advantage of Rust
auto-verification tools which base on Rust's strong type system
\cite{Verus, Prusti, Creusot, Flux}. These tool allows developers to annotate
Rust code with specifications, invariants and assertions and then verify them
formally, mathematically proving that the code satisfies the specifications.

This thesis demonstrated how unsafe raw pointer accesses are protected through
isolating memory regions,
leveraging language features like automatic bound checks for array types, and
type constructor that checks for their arguments. Furthermore, these features
can be used even more extensively to achieve a more fine-grained memory region
isolation, such as separating the NPT pools into the first level NPT pool,
second level NPT pool, and so on, or splitting the \rustcore{} metadata region
into multiple isolated regions, such as \code{VMInfo} region, \code{PMemInfo}
region, etc.
%In addition, the lifetime feature of Rust remains unexplored in our work.
%In contrast to the spatial protection achieved through memory region isolation,
%there are opportunities to utilize lifetimes to enforce temporal safety
%guarantees, such as preventing the usage of a reference before another
%reference has gone out of scope. Leveraging lifetimes in this manner allows
%for the implementation of lock order enforcement, among other possibilities.
%This approach is separate from our spatial protection and holds the potential
%to further enhance the safety of our code.

%Other low-level system software may also benefit from a Rust rewrite.
%
%write other stuff in Rust as well
%  CCA
%  pkvm intel
%  other parts of the kernel
%  Rust for Linux

