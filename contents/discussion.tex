% !TeX root = ../main.tex

\chapter{Discussion}
\label{sec:discussion}

Although we hope that Rust can help us with reducing the effort needed to secure
software, it is challenging to solely rely on Rust to provide safety guarantees
comparable to formal verification. Rust does not model hardware features
such as memory translation done by the Memory Management Unit (MMU), and
exception levels. Consequently, when building software on top of these features,
the compiler cannot check for misconfiguration or logical errors.
Moreover, \rustsec{}'s TCB \rustcore{} is not guaranteed to be memory safe either,
as there are still portions of unsafe Rust code
within Rcore that the compiler cannot ensure memory safety for.

Aside from memory safety, formal verification can also guarantee correctness in
terms of functionality.
In contrast, Rust alone cannot guarantee functional correctness. While Rust
autoverification tools introduced in \autoref{sec:fw} may offer some assistance
in this regard, it is clear that they do not match the level of assurance
offered by formal verification.
Nonetheless, conducting extensive testing may be a viable option to gain
confidence in ensuring the correct behavior of \rustcore{}.
