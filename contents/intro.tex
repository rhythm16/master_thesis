% !TeX root = ../main.tex

\chapter{Introduction}

Hypervisors are essential to cloud computing. They manage the hardware
resources to provide the virtual machine (VMs) abstraction and host
these VMs in the cloud.
The widely used commodity
hypervisors, such as KVM~\cite{kivity07kvm} or Hyper-V~\cite{hyperv},
include a large and complex TCB to satisfy users' requirements in
performance and functionality. These hypervisors were written in unsafe
languages like C, making them vulnerable to safety bugs, such as
out-of-bound memory access and use-after-free. For example, KVM
integrates an entire Linux OS kernel inside its TCB. Attackers that
successfully exploit hypervisor vulnerabilities may gain the ability
to steal or modify secret VM data.

Previous work~\cite{hypsec} has retrofitted commodity hypervisors into a small
trusted core that enforces resource access control to ensure the
confidentiality and integrity of VM data against hypervisor and host operating
system exploits. However, the security of the whole system still depends on the
implementation of the small trusted TCB. Any vulnerability in the trusted TCB
can void the guarantees of VM data confidentiality and integrity.
While~\cite{sekvm} extended the work of~\cite{hypsec} by formally verifying the
smaller TCB, the approach is not scalable since all code modifications
including the addition of new features, or code refactoring, requires a new
proof.

Rust is an emerging programming language that ensures strong memory safety
guarantees at compile time while offering performance efficiency.
Its distinctive ownership and lifetime system
effectively addresses potential safety issues that programmers may encounter.
Further, similar to programming languages like C, Rust allows developers to
directly manage low-level systems resources such as memory. Due to these
attributes, various previous work has adopted Rust to implement systems
software with critical security and performance requirements, including
operating systems~\cite{NrOS, Redleaf, TockOS, theseus},
hypervisors~\cite{DuVisor, RustyHermit}, web browsers~\cite{servo},
and TEEs~\cite{rustsgx,rustee}.
There has been recent adoption of Rust in the mainline Linux kernel. However,
instead of replacing the existing Linux kernel code written in C with Rust,
the current efforts were limited to developing new Rust-based device drivers.

Our work leverages the Rust programming language and migrate SeKVM \cite{sekvm},
a secure Linux KVM hypervisor from C to Rust, so that the resulting hypervisor
benefits from the strong safety guarantees that Rust automatically provides.
Our implementation, \rustsec{}, incorporates a small TCB called \rustcore{} to
protect VM confidentiality and integrity against the large and untrusted
hypervisor codebase that encompasses KVM’s host Linux kernel.
We identified and overcame the challenges that arose when trying to
incorporate a Rust TCB inside Linux \todo{TODO: elaborate}.
We also redesigned the \rustcore{} TCB to minimize the amount of unsafe Rust,
and enclosed the unsafe code within a safe abstraction and exposed a safe API
in order to implement complex functionalities in safe Rust, including CPU,
memory, VM boot protection, VM exit, and hypercall handlers.
Further, Rust’s type system is leveraged to ensure spatial memory safety of
\rustcore{}’s memory accesses by dividing physical memory into multiple
disjoint regions and guaranteeing all memory accesses done by Rcore is located
in the predefined regions.
This is achieved with customized Rust types for each memory region that
enforces bound-check to accesses, and mandating that \rustcore{} access a
memory region via each corresponding type.

% describe the benefits
\rustsec{} is the first secure Linux/KVM hypervisor written in Rust.
We spent less than one person year migrating SeKVM into \rustsec{}.
By migrating a C-based hypervisor to a Rust-based implementation,
we shift the responsibility of human auditing to the compiler.
This results in safer code and a more straightforward development process.
Performance evaluation of \rustsec{} on real Arm64 hardware shows that
\rustsec{} incurs modest performance overhead to application workloads
compared to mainline KVM and SeKVM. We demonstrate the practicality of
securing an existing commodity hypervisor by a C-to-Rust migration.

The rest of the paper will be organized as follows. Background and related work
will be discussed in \autoref{sec:bgrw}. The migration process of \rustsec{}
and the techniques used are described in \autoref{sec:migration}.
\autoref{sec:securercore} presents how we utilize Rust's safety features to
design and secure \rustcore{} memory accesses.
Evaluation of \rustsec{} and its comparison with mainline KVM and SeKVM is
covered in \autoref{sec:eval}. At last, we conclude the paper in
\autoref{sec:conclusions}.
