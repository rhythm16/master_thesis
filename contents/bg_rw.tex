% !TeX root = ../main.tex
%\raggedbottom

\chapter{Background and Related Work}
\label{sec:bgrw}

\section{SeKVM}
%\todo{TODO: WIP, and introduce \rustsec{} design here.}
%Various previous work~\cite{lisosp21,zhang2011cloudvisor,zeyu20usenix,pkvm,fidelius-hpca18,sekvm}
%redesigned the hypervisor to protect VMs.
Our work explores the possibility of rewriting a hypervisor TCB in Rust, we
started from the implementation of SeKVM \cite{sekvm}.
SeKVM leveraged an earlier design~\cite{hypsec} to retrofit and secure KVM,
it relies on a small TCB called \secore{} to protect VM confidentiality and
integrity against an untrusted KVM host that encompasses the host Linux kernel
integrated with KVM.
To reduce the TCB,
the design separates access control from resource allocation. \secore{}
has full access to hardware resources to perform access control to
protect VM data. SeKVM assumes VMs employ an end-to-end approach to
encrypt I/O data. Since VMs already protects their I/O data, SeKVM
focuses on protecting VM data in CPU and memory.
The KVM host provides device drivers and complex virtualization features
such as resource allocation, scheduling, and VM management. \secore{} is
responsible for protecting VM data, ensuring VM CPU registers and memory
allocated to the VM are inaccessible to the untrusted KVM host.

\secore{} leverages hardware virtualization
support to deprivilege the KVM host at a lower privileged level, ensuring
the untrusted host cannot disable or control privileged hardware
features. \secore{} enables the nested page tables (NPT) when running
the KVM host and VMs so that they do not have direct access to physical
memory.
\secore{} unmaps its own private memory pages from the respective NPTs,
making them inaccessible
to VMs and the host. \secore{} unmaps a given VM's memory pages from
the KVM host's or other VMs' NPTs to isolate these pages.
\secore{} allocates NPTs for the KVM host
and VMs from its memory pool, to which the host and VMs have no access.
Since VM and \secore{} memory is unmapped from the host NPT, a
compromised host that accesses these memory pages causes an NPT fault
that traps to \secore{}. \secore{} routes NPT faults to itself,
allowing it to reject invalid host memory accesses.

SeKVM reuses the device drivers from the KVM host to manage
I/O devices and provide I/O virtualization. An attacker from the host
can control devices to perform Direct Memory Access (DMA) to
read or write VM memory. To protect VM memory against such DMA
attacks, SeKVM leverages IOMMU to restrict all devices'
memory accesses. \secore{} allocates and manages IOMMU page
tables from its private memory. \secore{} trap-and-emulates the
KVM host's access to the IOMMU and manages the IOMMU page tables
for each DMA-capable device.

\secore{} runs in a higher privileged CPU mode than the host to
interpose VM exits and interrupts, ensuring the host cannot compromise
VM data.
VM exit handling may require the KVM host's functionality. Before
entering the host, \secore{} saves VM CPU registers from the
hardware to its private memory then restores the host's CPU registers to
the hardware; vice versa is done before entering the VM. An attacker has
no access \secore{}'s private memory, and thus, the VM CPU registers.

SeKVM leverages Arm's hardware Virtualization Extensions (VE) to
simplify its implementation. \secore{} runs in the hypervisor mode
(EL2) to control Arm VE features to deprivilege the KVM host in a less
privileged kernel mode (EL1). \secore{} uses Arm's NPTs, stage 2
page tables to enforce memory access control. Stage 2 page table
translation only affects the software running in Arm's kernel and user
(EL0) mode, but not software running in EL2. \secore{} employs an
identity map in the KVM host's stage 2 page tables, translating each
host machine's physical addresses (hPA) to an identical hPA. This
allows SeKVM to reuse Linux's memory allocator to manage memory
implicitly. Running in EL2 allows \secore{} to isolate itself
from the host and VMs. First, running in EL2 isolates \secore{} in
a separate address space from EL1 and EL0. Second, Arm provides banked
system registers to EL1 and EL2. The host cannot access
EL2 registers to disable \secore{}'s protection. For instance,
the host cannot modify the register VTTBR\_EL2, which stores the base
address of the stage 2 page tables. \secore{} exposes a set of
required hypercalls~\cite{hypsec} for the host to request services
that require EL2 privileges. For example, once the host schedules
a virtual CPU, it makes a hypercall to \secore{} to context switch to
the VM. \secore{} leverages Arm's IOMMU, the System Memory
Management Unit (SMMU)~\cite{smmu-whitepaper}, to protect against
DMA attacks.

%Unlike our work, none of them used
%Rust to secure their hypervisor implementation.
%\rustsec{} and SeKVM~\cite{sekvm} both leveraged an earlier design~\cite{hypsec}
%to retrofit and secure KVM, providing the same level of VM protection. SeKVM
%included a formally verified core to protect VMs against an untrusted host Linux
%kernel, while \rustsec{} relies on a Rust-based \rustcore{} to protect VMs.
%Formal verification of the concurrent C-based SeKVM core requires significant effort.
%The authors took two person-years to complete the correctness and security proofs.
%In contrast, our Rust-based implementation took less than one person-year while
%ensuring properties verified systems provide, including memory safety,
%data race, and deadlock freedom.

\section{The Rust Programming Language}
Rust, compared to C,
is a relatively young programming language aiming to be safe and fast.
It enables programs to be memory-safe without requiring programmers to
painstakingly manage memory, as in traditional languages (e.g., C/C++).
Unlike other memory-safe languages (e.g., Python, Go, etc.),
Rust does not leverage garbage collection mechanisms
to ensure memory safety.
Instead, it introduces the concepts of lifetimes and ownership
to mandate the programmer to follow specific rules.
This paradigm of statically enforcing programming rules
empowers Rust to perform comparably to C since Rust's compiler has complete
control over the code that
runs during runtime and can optimize it accordingly.
Additionally, Rust's safety rules ensure that
no memory safety bugs will be present when satisfied,
and the compiler automatically checks and prevents any violation of these rules.

\textbf{Ownership and Lifetimes.}
In Rust, each piece of data is said to be \textit{owned} by a single
variable, and it is automatically \textit{dropped} (freed) when the
variable's \textit{lifetime} ends. A variable's lifetime ends as the program
control flow exits the block in which the variable is declared.
In \autoref{lst:lifetime}, \code{y}'s lifetime starts at line 5 and ends at
line 7 as the block closes. Hence, the \code{println!} macro is unable to find
the value \code{y}, whose lifetime has already ended.
Ownership can be transferred or \textit{moved}. For example,
assigning the owning variable to a new variable moves the ownership of the
data to the new variable. And passing the variable into a function also moves
the data ownership into the function.
In both situations, the original variable returns to the uninitialized state,
and using it would result in a compilation error.

\begin{listing}[hbtp]
    \begin{minted}{rust}
// this code sample does *not* compile
{
  let x = 1;
  {            // create new scope
    let y;
    y = x;
  }            // y is dropped

  // compilation error, y's lifetime has ended
  println!("The value of 'y' is {}.", y);
}
    \end{minted}
    \caption{Rust lifetime example}
    \label{lst:lifetime}
    \vspace{-0.2cm}
\end{listing}

\textbf{Borrowing.}
Ownership lacks the flexibility of argument passing.
Rust addresses this by \textit{borrowing},
a mechanism that allows accessing data without gaining ownership.
A variable can borrow ownership from another variable to acquire a
\textit{reference} to the data. References can be divided into two categories,\textit{shared}
references and \textit{exclusive} references. The reference can only be read
and not modified with a shared reference. Nevertheless, multiple shared
references for a specific value can be held simultaneously.
On the other hand, exclusive references allow reading from and modifying the
value. However, having any other kind of reference active simultaneously for
that value is not permitted.

In summary, Rust's borrowing rule enforces \textit{aliasing xor mutability}
meaning there can be multiple shared references or a single exclusive
reference.
In \autoref{lst:borrowingrule}, line 6 would not compile because it tries to create
a mutable reference (\code{z}) to \code{x}, while \code{y} already borrowed
\code{x} immutably. \code{y}'s lifetime ends on line 8 as it gets used for the
last time; therefore \code{z} can be created on line 10 and used on line 11.
However, if line 13 is uncommented, \code{y}'s lifetime would be extended to
line 13, making the creation of \code{z} on line 10 break the borrowing rules.

\begin{listing}[hbtp]
    \begin{minted}{rust}
{
  let mut x = vec![1, 2, 3];
  let y = &x; // immutable borrow of x

  // this line would fail to compile because x is already borrowed immutably by y
  /* let z = &mut x; */

  println!("x = {:?}", x); // This line works
  println!("y = {:?}", y); // This line works

  let z = &mut x; // mutable borrow of x
  z.push(4);

  // this line would fail to compile because x is borrowed mutably by z
  /* println!("y = {:?}", y); */
}
    \end{minted}
    \caption{Rust enforces \textit{aliasing xor mutability}}
    \label{lst:borrowingrule}
    \vspace{-0.2cm}
\end{listing}

\textbf{unsafe Rust.}
Rust's safety checks are sometimes too restrictive regarding tasks like
low-level hardware access or special optimizations. These operations are
inherently unsafe and hence impossible to follow the rules mandated by Rust.
However, they are still necessary for low-level software such as
hypervisors. To provide flexibility for these operations, Rust allows
parts of the program to opt out of its safety checks via the \textit{unsafe}
keyword. Traits, functions, and code blocks can be marked as unsafe to disable
the checks that the compiler would normally enforce. However, using unsafe code
also means that the responsibility for ensuring memory safety is shifted from
the compiler to the programmer. Therefore, it is crucial to exercise
caution when using unsafe code to avoid introducing bugs or security
vulnerabilities.

\textbf{Interior unsafe.}
While most low-level code is written in unsafe code, Rust introduces
the concept of \textit{interior unsafe}~\cite{ruststudy}. A function is considered
interior unsafe if it exposes a safe interface but contains unsafe blocks
in implementation. This allows unsafe operations to be encapsulated
into safe abstractions. For instance, in
\autoref{lst:interiorunsafe}, Rust's \code{replace} function can be
called by safe Rust, but it is implemented using unsafe raw pointer operations.
At line 6, \code{ptr::read} is used to copy a bit-wise value from \code{dest}
into \code{result} without moving it, and at line 7, \code{ptr::write} overwrites
the memory location pointed to by \code{dest} with the given value \code{src}
without reading or dropping the old value. Lastly, at line 8, \code{result} is
returned to the function's caller.

\begin{listing}[hbtp]
    \begin{minted}{rust}
pub const fn replace<T>(dest: &mut T, src: T) -> T {
  // SAFETY: We read from `dest` but directly write `src` into it afterward,
  // such that the old value is not duplicated. Nothing is dropped and
  // nothing here can panic.
  unsafe {
    let result = ptr::read(dest);
    ptr::write(dest, src);
    result
  }
}
    \end{minted}
    \caption{interior unsafe in Rust's \code{replace} function}
    \label{lst:interiorunsafe}
    \vspace{-0.2cm}
\end{listing}

This leads to a design practice that interior unsafe functions should provide
the necessary checks that prevent the unsafe code from producing any undefined
behavior or memory safety bugs.
The callee in the safe world hence bears no responsibility to ensure safety.

\textbf{Interior Mutability.}
Mutating the underlying data via an immutable reference
is forbidden in Rust.
However, this might be too restrictive for
implementing efficient algorithms or data structures.
For example, programmers might want to add a cache
in a read-only search data structure
to optimize the search time.
Nevertheless, updating the state of the cache implies
the need for mutability,
which violates the read-only constraint.
Hence, we need the ability to mutate states even under a read-only scene.
To address this issue,
the Rust standard library provides some special types
that can mutate the underlying data
even if we only have read-only access to the data holder.
This design pattern is known as Interior Mutability.
Implementing these types requires \code{unsafe} operations
to bend Rust's usual rules that govern mutation and borrowing.
To avoid violating the virtue of the Rust safety assumptions,
these types ensure that the borrowing rules,
i.e., one mutable borrower at the same time
and no mutable borrowers when read-only borrowers exist,
will be followed at runtime.
If these rules get violated,
the implementation of these types has the responsibility
to stop the behavior to avoid safety issues.
For example, the type \code{Mutex} in Rust is a type that provides
interior mutability.
It uses a lock to ensure that
only one borrower of the inner data can appear simultaneously
to enforce safe mutation of the inner data, even without explicit mutability to \code{Mutex}.
More precisely, when attempting to borrow data that has already been borrowed,
the \code{Mutex} enforces a busy wait until the data is returned,
thereby allowing only one borrower at a time.
However, if a thread borrows the inner data of \code{Mutex} while it is already borrowing it,
\code{Mutex} will wait forever, i.e., result in a self-deadlock.
\footnote{This is the behavior when using \texttt{\textcolor{gray}Mutex} on Linux.
On Windows, \texttt{\textcolor{gray}Mutex} might panic.}

\textbf{Generics and Traits.}
In addition to the safety mechanisms, Rust, as a modern programming language,
provides handy features to make programming easier.
\code{Generic} allows code to work with type parameters,
reducing the effort of writing similar code for multiple types.
For example, using generics, the \code{Mutex} type can
hold and lock any arbitrary type.
Rust traits are properties or interfaces that can be implemented on types; traits typically require the implementing type
to supply function implementations for its trait methods.
Additionally, combined with \code{Generic},
a trait can be treated as a restriction on type specifications such as function arguments or struct fields.
The restriction is called a \textit{trait bound}.
For example, the \code{Clone} trait requires the implementing type
to provide implementations for its \code{clone} and \code{clone\_from} functions
to make copies of themselves.
A \code{Generic} function or type can use a trait bound to
require its type argument to implement \code{Clone},
so that it can invoke the \code{clone} function that the argument implements.

\textbf{Graceful Error Handling.}
Rust offers a graceful approach to error handling.
The \code{Result<T, E>} and \code{Option<T>} types in Rust
explicitly admit the possibility of errors.
\code{Re\-su\-lt} represents the outcome as either \code{Ok(T)} or \code{Err(E)},
with \code{T} denoting the desired result
and \code{E} representing the error reason.
Programmers are obligated to handle \code{E} when accessing \code{T},
and not doing so would result in a compilation error.
To simplify error handling,
Rust provides a convenient syntactic sugar,
the \code{?} operator.
It permits the retrieval of \code{T} from \code{Result} if it is \code{Ok(T)},
or early return of \code{E} if it is \code{Err(E)}.
Similarly, \code{Option} simplifies error handling with two possibilities:
\code{Some(T)} or \code{None}, where \code{None} can signify a trivial error.
These types prevent unexpected errors
when accessing a potentially non-existing value in the program.

\textbf{Copy and Drop Traits.}
Some traits in Rust have intrinsic meaning to the compiler. For example, the
\code{Drop} trait tells the compiler that a type has special freeing code, and
the \code{Drop} trait's \code{drop} function should be invoked when an instance
of the type goes out of scope. And the \code{Copy} trait, when implemented for
a type indicates that the type should be byte-by-byte copied when the
assignment (\code{=}) operator is used instead of Rust's typical semantic
of moving the ownership to the new variable. Interestingly, Rust forbids a type
from being \code{Drop} and \code{Copy} simultaneously, the designers of the
language observed that if a type requires special deallocating code (the
\code{drop} function), then it should also require a special copying function,
rather than just copying it byte-by-byte. For instance, a type that holds a
reference to the heap requires a \code{drop} function that frees the data
pointed to by the reference, copying the object of the type in a byte-by-byte
manner introduces risks of double-free, use-after-free, etc.

